\apendice{Software Requirement Specification}

\section{Introduction}
The following chapter provides a specification of the software’s overall requirements. Showing its functions, limitations and user interaction, the catalogue can be considered as a contract between clients and developers. Aside from giving textual descriptions, use case diagrams help to clarify the requirements and interactions in more detail.


\section{General Objectives}
The following steps and objectives are related to the software's requirements:
\begin{itemize}
\item Design of a conversational interface: the conversation flow between user and chatbot is mapped to the natural language processing platform api.ai which then parses the user input into formalized data. The parsed input is interpreted by the chatbot and triggers the desired behaviour, such as recommendation or storage of important user information.

\item A recommender system must be implemented to provide personalized tourist recommendations. The recommender is based on data the user has shared with the chatbot and additional data of similar users. To overcome the problem of initially sparse user data, different recommendation methods are combined as well as retrieving existing user data from other sources and/or generating data.

\item Extracting tourist data from a spatial information database based on OpenStreetMap: the data is filtered so that only data of touristic importance is evaluated by the recommender and presented to the user.
\end{itemize}

\section{Software Requirement Catalogue}
This section outlines the software requirements. At first, the participating actors are introduced. Then, the requirements are examined, dividing them into functional and non-functional requirements. 

\subsection{Participating actors}
This software has only one main actor, the user. Using the Telegram messenger, users interact with the chatbot. There are no different user roles, so each user has the same range of features to use. 

\subsection{Functional Requirements}
	This section describes the software’s full range of features:
	
\begin{itemize}
\item Using the Telegram interface, messages to the chatbot can be introduced which are answered accordingly.
\item Using the Telegram interface, a location can be introduced which is used as the basis for the chatbot recommendations.
\item The proximity of the recommendations can be specified by the user. If no radius is entered, the default value of 1 km is used as a distance between the user location and examined point of interests.
\item The chatbot provides a recommended point of interest within the chosen proximity. The recommendation result contains the name, location, point of interest category as well as an additional picture retrieved from Foursquare (if available).
\item Recommended points of interests are rated by the users. The rating is saved and used as a basis for future recommendations.
\item The user can access the information the chatbot has collected about him, such as already recommended points of interests or saved interests.
\item To refine recommendations, user chat messages are evaluated using natural language processing and saved in the user profile.
\end{itemize}

\subsection{Non-functional Requirements}
This section describes the so-called non-functional requirements which contain technical as well as operational requirements.

\begin{itemize}
\item To use the chatbot, the Telegram messenger app has to be installed on a smart device (smartphones or tablets). Although Telegram is also available as a desktop application, these versions do not support sending locations and are therefore not suitable.
\item The device must be connected to the internet to use the chatbot.
\item The device must be capable of receiving GPS information to calculate its current location. 
\item The conversational interface should be intuitive, so the user is able to communicate with the chatbot without previously reading an exhaustive tutorial. To facilitate user decisions, mutually exclusive keyboard buttons are used.
\item The chatbot should be able to handle user requests adequately. Questions and demands concerning travelling should be understood and answered satisfyingly. Other requests are rejected politely.
\end{itemize}

\subsection{Limitations}
There are multiple limitations present due to the fact that the software can be still considered as a prototype. In a future enhancement, most of these limitations should be remedied.
\begin{itemize}
\item The used OpenStreetMap data was downloaded once and then used offline. To keep the data up-to-date, an automatic update mechanism should be set up.
\item Due to performance reasons and sparse user rating data, the prototype only gives recommendations for the city of \textit{Barcelona}, Spain. In a future enhancement, a bigger OSM region should be covered.
\end{itemize}
\pagebreak

\section{Requirement Specification}
\subsection{Use Case Diagrams}
\subsubsection{General Use Case Diagram}
\imagen{general_use_cases.png}{General Use Cases}

	The shown use case diagram contains the six main interactions the user performs using the chatbot. The use case “Help” is used to show the user the main features of the chatbot. In “Chat about Preferences” the chatbot collects information about the user by chatting with him. This data is used to complete the user profile for recommendations. On the other hand, “Rate Recommendation” shows how the user rates previously recommended points of interests.
	
	The use cases “Show Past Recommendations” and “Show User Information” both aim to provide the user an understanding of the data the chatbot has already collected of the user.
	
The use case “Get Recommendation” represents the interactions between user and chatbot that lead to the delivery of user adapted recommendations. Because of its complexity, this use case is shown in more detail in the following section.

\subsubsection{Use Case Diagram - "Get Recommendation"}
\imagen{recommendation_use_case.png}{Use Case - "Get Recommendation"}

This use case shows the involved components in providing the user with recommendations. In order to get recommendations, the user is able to specify a recommendation radius to set the maximal distance between himself and the point of interest. This step is optional. After showing the user the recommended point of interest, he is asked to give a first impression of the point of interest in order to refine future recommendations.

\subsection{Use Case Templates}
The previously defined use cases are explained in the following templates, showing the circumstances in which a use case is handled. Additionally to the given description, the flow of events will be shown in detail in the Design Specification, illustrated by sequence diagrams.

\begin{table}[]
\centering
\begin{tabular}{|p{3.5cm}|p{1.5cm}|p{6cm}|}
\hline
\rowcolor[HTML]{C0C0C0} 
\textbf{Use Case ID} & \multicolumn{2}{l|}{\cellcolor[HTML]{C0C0C0}\textbf{UC-01}} \\ \hline
Use Case Name & \multicolumn{2}{l|}{Chat About Preferences} \\ \hline
Description & \multicolumn{2}{l|}{\begin{tabular}[c]{@{}p{7.5cm}@{}}The user chats with the chatbot about his preferences.\end{tabular}} \\ \hline
Trigger  & \multicolumn{2}{l|}{\begin{tabular}[c]{@{}p{7.5cm}@{}}User connects to the chatbot for the first time or shows proactively the intention to chat about his preferences.\end{tabular}} \\ \hline
Precondition & \multicolumn{2}{l|}{\begin{tabular}[c]{@{}p{7.5cm}@{}}The chatbot is in a state in which the user is allowed to type messages independently, meaning that the user does not conduct another predefined conversation.\end{tabular}} \\ \hline
 & Step & Action \\ \cline{2-3} 
 & 1 & \begin{tabular}[c]{@{}p{6cm}@{}}Chatbot: “So, tell me, what are you interested in when you visit a new place?”\end{tabular}  \\ \cline{2-3} 
 & 2 & User answers and interest is filtered  \\ \cline{2-3} 
\multirow{-4}{*}{Flow of Events} & 3 & \begin{tabular}[c]{@{}p{6cm}@{}}The interest is saved and the user is told to repeat the process whenever he likes.\end{tabular}\\ \hline
  & Step & Action  \\ \cline{2-3} 
\multirow{-2}{*}{Alternate Flow} & 3a & User response is not understood, so the user is asked to rephrase his answer.  \\ \hline
Postcondition & \multicolumn{2}{l|}{\begin{tabular}[c]{@{}p{7.5cm}@{}}The chatbot is in a state in which the user is allowed to type messages independently.\\ \\ The user profile is updated with user interests.\end{tabular}} \\ \hline
  & Step & Action \\ \cline{2-3} 
\multirow{-2}{*}{Exceptions}  & 2 & User answers “No”, so the use case is aborted. \\ \hline
Frequency of Use & \multicolumn{2}{l|}{Medium} \\ \hline
Importance & \multicolumn{2}{l|}{High}  \\ \hline
\end{tabular}
\caption{UC-01 - Chat About Preferences}
\label{chatusecase}
\end{table}

\begin{table}[]
\centering
\begin{tabular}{|p{3.5cm}|p{1.5cm}|p{6cm}|}
\hline
\rowcolor[HTML]{C0C0C0} 
\textbf{Use Case ID} & \multicolumn{2}{l|}{\cellcolor[HTML]{C0C0C0}\textbf{UC-02}} \\ \hline
Use Case Name & \multicolumn{2}{l|}{Get Recommendation} \\ \hline
Description & \multicolumn{2}{l|}{\begin{tabular}[c]{@{}p{7.5cm}@{}}The user asks for a recommended point of interest. Based on the user information, the recommender returns points of interests close to the user. \end{tabular}} \\ \hline
Trigger  & \multicolumn{2}{l|}{\begin{tabular}[c]{@{}p{7.5cm}@{}}The user asks for a recommendation.\end{tabular}} \\ \hline
Precondition & \multicolumn{2}{l|}{\begin{tabular}[c]{@{}p{7.5cm}@{}}The chatbot is in a state in which the user is allowed to type messages independently, meaning that the user does not conduct another predefined conversation.\end{tabular}} \\ \hline
 & Step & Action \\ \cline{2-3} 
 & 1 & \begin{tabular}[c]{@{}p{6cm}@{}}Chatbot asks for the user's current location.\end{tabular}  \\ \cline{2-3} 
 & 2 & User enters his location  \\ \cline{2-3} 
 & 3 & Chatbot presents user a recommended point of interest.  \\ \cline{2-3} 
 & 4 & Use Case $\rightarrow$ Rate First Impression (see \ref{ratefirstusecase}) \\ \cline{2-3} 
 & 5 & Chatbot: “Do you want to see another recommendation?”  \\ \cline{2-3} 
\multirow{-7}{*}{Flow of Events} & 6 & \begin{tabular}[c]{@{}p{6cm}@{}}User enters No.\end{tabular}\\ \hline
 & Step & Action  \\ \cline{2-3}
 & 1a & User starts recommendation process by entering its current location which is followed by step 3.  \\ \cline{2-3} 
\multirow{-3}{*}{Alternate Flow}& 6a & User enters Yes, so steps 3-5 are repeated. \\ \hline
Postcondition & \multicolumn{2}{l|}{\begin{tabular}[c]{@{}p{7.5cm}@{}}The chatbot is in a state in which the user is allowed to type messages independently.\\ \\ Recommended points of interests the user liked are saved and marked as unrated.\end{tabular}} \\ \hline
  & Step & Action \\ \cline{2-3} 
\multirow{-2}{*}{Exceptions}  & 3 & Chatbot does not find any (more) points of interests for the user and cancels recommendation process. \\ \hline
Frequency of Use & \multicolumn{2}{l|}{High} \\ \hline
Importance & \multicolumn{2}{l|}{High}  \\ \hline
\end{tabular}
\caption{UC-02 - Get Recommendation}
\label{recommendusecase}
\end{table}

\begin{table}[]
\centering
\begin{tabular}{|p{3.5cm}|p{1.5cm}|p{6cm}|}
\hline
\rowcolor[HTML]{C0C0C0} 
\textbf{Use Case ID} & \multicolumn{2}{l|}{\cellcolor[HTML]{C0C0C0}\textbf{UC-03}} \\ \hline
Use Case Name & \multicolumn{2}{l|}{Rate Recommendation} \\ \hline
Description & \multicolumn{2}{l|}{\begin{tabular}[c]{@{}p{7.5cm}@{}}The user rates a points of interest that was previously recommended to him. \end{tabular}} \\ \hline
Trigger  & \multicolumn{2}{l|}{\begin{tabular}[c]{@{}p{7.5cm}@{}}The user greets the chatbot or wants to see past recommendations and has unrated recommendations.\end{tabular}} \\ \hline
Precondition & \multicolumn{2}{l|}{\begin{tabular}[c]{@{}p{7.5cm}@{}}The user was given a recommendation before that he has not rated yet. \end{tabular}} \\ \hline
 & Step & Action \\ \cline{2-3} 
 & 1 & \begin{tabular}[c]{@{}p{6cm}@{}}Chatbot asks how the user liked the first unrated recommended point of interest.\end{tabular}  \\ \cline{2-3} 
 & 2 & User chooses from mutually exclusive rating buttons (e.g. 1 stars to 5 stars rating)  \\ \cline{2-3} 
\multirow{-4}{*}{Flow of Events} & 3 & \begin{tabular}[c]{@{}p{6cm}@{}}Chatbot: \textit{“Thanks for the rating!”}\end{tabular}\\ \hline
 & Step & Action  \\ \cline{2-3}
\multirow{-1}{*}{Alternate Flow}& - & - \\ \hline
Postcondition & \multicolumn{2}{l|}{\begin{tabular}[c]{@{}p{7.5cm}@{}}The chatbot is in a state in which the user is allowed to type messages independently.\\ \\ The maximal radius for that user is saved and used for the next recommendations.\end{tabular}} \\ \hline
  & Step & Action \\ \cline{2-3} 
\multirow{-2}{*}{Exceptions}  & 2 & The user states he does not want to rate the point of interest, so the use case is aborted. \\ \hline
Frequency of Use & \multicolumn{2}{l|}{Medium} \\ \hline
Importance & \multicolumn{2}{l|}{High}  \\ \hline
\end{tabular}
\caption{UC-03 - Rate Recommendation}
\label{rateusecase}
\end{table}

\begin{table}[]
\centering
\begin{tabular}{|p{3.5cm}|p{1.5cm}|p{6cm}|}
\hline
\rowcolor[HTML]{C0C0C0} 
\textbf{Use Case ID} & \multicolumn{2}{l|}{\cellcolor[HTML]{C0C0C0}\textbf{UC-04}} \\ \hline
Use Case Name & \multicolumn{2}{l|}{Specify Recommendation Radius} \\ \hline
Description & \multicolumn{2}{l|}{\begin{tabular}[c]{@{}p{7.5cm}@{}}The user sets the radius in which he wants the recommended points of interests to be in.  \end{tabular}} \\ \hline
Trigger  & \multicolumn{2}{l|}{\begin{tabular}[c]{@{}p{7.5cm}@{}}The user asks to set the recommendation distance.\end{tabular}} \\ \hline
Precondition & \multicolumn{2}{l|}{\begin{tabular}[c]{@{}p{7.5cm}@{}}The chatbot is in a state in which the user is allowed to type messages independently, meaning that the user does not conduct another predefined conversation.\end{tabular}} \\ \hline
 & Step & Action \\ \cline{2-3} 
 & 1 & \begin{tabular}[c]{@{}p{6cm}@{}}Chatbot asks the user about his preferred maximal recommendation radius.\end{tabular}  \\ \cline{2-3} 
 & 2 & User answers with a positive numeric value.  \\ \cline{2-3} 
\multirow{-4}{*}{Flow of Events} & 3 & \begin{tabular}[c]{@{}p{6cm}@{}}Chatbot repeats: \textit{“Fine, I set the maximal radius to (repeat value)”}.\end{tabular}\\ \hline
 & Step & Action  \\ \cline{2-3}
\multirow{-1}{*}{Alternate Flow}& - & - \\ \hline
Postcondition & \multicolumn{2}{l|}{\begin{tabular}[c]{@{}p{7.5cm}@{}}The chatbot is in a state in which the user is allowed to type messages independently.\\ \\ The rating is saved in the ratings file and the corresponding recommendation is marked as rated.\end{tabular}} \\ \hline
  & Step & Action \\ \cline{2-3} 
\multirow{-2}{*}{Exceptions}  & 1 & User doesn’t answer with a positive numeric value, so the chatbot asks again or aborts the use case. \\ \hline
Frequency of Use & \multicolumn{2}{l|}{Low} \\ \hline
Importance & \multicolumn{2}{l|}{Medium}  \\ \hline
\end{tabular}
\caption{UC-04 - Specify Recommendation Radius}
\label{recradiususecase}
\end{table}

\begin{table}[]
\centering
\begin{tabular}{|p{3.5cm}|p{1.5cm}|p{6cm}|}
\hline
\rowcolor[HTML]{C0C0C0} 
\textbf{Use Case ID} & \multicolumn{2}{l|}{\cellcolor[HTML]{C0C0C0}\textbf{UC-05}} \\ \hline
Use Case Name & \multicolumn{2}{l|}{Rate First Impression} \\ \hline
Description & \multicolumn{2}{l|}{\begin{tabular}[c]{@{}p{7.5cm}@{}}When the user is given a recommended point of interest, he is immediately asked of his first impression to refine future recommendations.\end{tabular}} \\ \hline
Trigger  & \multicolumn{2}{l|}{-} \\ \hline
Precondition & \multicolumn{2}{l|}{\begin{tabular}[c]{@{}p{7.5cm}@{}}The user has just received a recommended point of interest (see \ref{recommendusecase})\end{tabular}} \\ \hline
 & Step & Action \\ \cline{2-3} 
 & 1 & \begin{tabular}[c]{@{}p{6cm}@{}}Chatbot asks the user about his first impression of the recommended point of interest.\end{tabular}  \\ \cline{2-3} 
\multirow{-3}{*}{Flow of Events} & 3 & \begin{tabular}[c]{@{}p{6cm}@{}}User chooses from mutually exclusive buttons (e.g. \textit{“Sounds good!”} or \textit{“Don’t like it”}).\end{tabular}\\ \hline
 & Step & Action  \\ \cline{2-3}
\multirow{-1}{*}{Alternate Flow}& - & - \\ \hline
Postcondition & \multicolumn{2}{l|}{\begin{tabular}[c]{@{}p{7.5cm}@{}}The chatbot is in a state in which the user is allowed to type messages independently.\\ \\ The rating is saved in the ratings file and the corresponding recommendation is marked as rated.\end{tabular}} \\ \hline
  & Step & Action \\ \cline{2-3} 
\multirow{-2}{*}{Exceptions}  & 1 & User does not answer accordingly, so the point of interest is discarded and no rating is saved.\\ \hline
Frequency of Use & \multicolumn{2}{l|}{High} \\ \hline
Importance & \multicolumn{2}{l|}{Medium}  \\ \hline
\end{tabular}
\caption{UC-05 - Rate First Impression}
\label{ratefirstusecase}
\end{table}

\begin{table}[]
\centering
\begin{tabular}{|p{3.5cm}|p{1.5cm}|p{6cm}|}
\hline
\rowcolor[HTML]{C0C0C0} 
\textbf{Use Case ID} & \multicolumn{2}{l|}{\cellcolor[HTML]{C0C0C0}\textbf{UC-06}} \\ \hline
Use Case Name & \multicolumn{2}{l|}{Show Past Recommendations} \\ \hline
Description & \multicolumn{2}{l|}{\begin{tabular}[c]{@{}p{7.5cm}@{}}The chatbot provides the user with information about the points of interests that were recommended to him previously.\end{tabular}} \\ \hline
Trigger  & \multicolumn{2}{l|}{\begin{tabular}[c]{@{}p{7.5cm}@{}}The user asks to see his previous recommendations.\end{tabular}} \\ \hline
Precondition & \multicolumn{2}{l|}{\begin{tabular}[c]{@{}p{7.5cm}@{}}The chatbot is in a state in which the user is allowed to type messages independently, meaning that the user does not conduct another predefined conversation.\end{tabular}} \\ \hline
 & Step & Action \\ \cline{2-3} 
\multirow{-2}{*}{Flow of Events} & 1 & \begin{tabular}[c]{@{}p{6cm}@{}}The chatbot lists the past recommendations the user was interested in.\end{tabular}\\ \hline
 & Step & Action  \\ \cline{2-3}
\multirow{-2}{*}{Alternate Flow}& 2 & When there are unrated items left, the user is asked to rate a previously recommended point of interest (see \ref{rateusecase}) \\ \hline
Postcondition & \multicolumn{2}{l|}{\begin{tabular}[c]{@{}p{7.5cm}@{}}The chatbot is in a state in which the user is allowed to type messages independently.\end{tabular}} \\ \hline
  & Step & Action \\ \cline{2-3} 
\multirow{-2}{*}{Exceptions}  & 1 & There are no past recommendations so far, so the user is told to ask for a recommendation first in order to use this feature.\\ \hline
Frequency of Use & \multicolumn{2}{l|}{Medium} \\ \hline
Importance & \multicolumn{2}{l|}{Medium}  \\ \hline
\end{tabular}
\caption{UC-06 - Show Past Recommendations}
\label{pastrecommendationsusecase}
\end{table}

\begin{table}[]
\centering
\begin{tabular}{|p{3.5cm}|p{1.5cm}|p{6cm}|}
\hline
\rowcolor[HTML]{C0C0C0} 
\textbf{Use Case ID} & \multicolumn{2}{l|}{\cellcolor[HTML]{C0C0C0}\textbf{UC-07}} \\ \hline
Use Case Name & \multicolumn{2}{l|}{Help} \\ \hline
Description & \multicolumn{2}{l|}{\begin{tabular}[c]{@{}p{7.5cm}@{}}The user asks for help and gets an overview of the chatbot features.\end{tabular}} \\ \hline
Trigger  & \multicolumn{2}{l|}{\begin{tabular}[c]{@{}p{7.5cm}@{}}The user asks for help.\end{tabular}} \\ \hline
Precondition & \multicolumn{2}{l|}{\begin{tabular}[c]{@{}p{7.5cm}@{}}The chatbot is in a state in which the user is allowed to type messages independently, meaning that the user does not conduct another predefined conversation.\end{tabular}} \\ \hline
 & Step & Action \\ \cline{2-3} 
\multirow{-2}{*}{Flow of Events} & 1 & \begin{tabular}[c]{@{}p{6cm}@{}}The chatbot gives an overview of the chatbot features (rating, chatting, recommendations).\end{tabular}\\ \hline
 & Step & Action  \\ \cline{2-3}
\multirow{-2}{*}{Alternate Flow}& - & - \\ \hline
Postcondition & \multicolumn{2}{l|}{\begin{tabular}[c]{@{}p{7.5cm}@{}}The chatbot is in a state in which the user is allowed to type messages independently.\end{tabular}} \\ \hline
  & Step & Action \\ \cline{2-3} 
\multirow{-2}{*}{Exceptions}  & - & - \\ \hline
Frequency of Use & \multicolumn{2}{l|}{Low} \\ \hline
Importance & \multicolumn{2}{l|}{Low}  \\ \hline
\end{tabular}
\caption{UC-07 - Help}
\label{helpusecase}
\end{table}

\begin{table}[]
\centering
\begin{tabular}{|p{3.5cm}|p{1.5cm}|p{6cm}|}
\hline
\rowcolor[HTML]{C0C0C0} 
\textbf{Use Case ID} & \multicolumn{2}{l|}{\cellcolor[HTML]{C0C0C0}\textbf{UC-08}} \\ \hline
Use Case Name & \multicolumn{2}{l|}{Show User Information} \\ \hline
Description & \multicolumn{2}{l|}{\begin{tabular}[c]{@{}p{7.5cm}@{}}The user is provided with the personal information the chatbot has collected so far.\end{tabular}} \\ \hline
Trigger  & \multicolumn{2}{l|}{\begin{tabular}[c]{@{}p{7.5cm}@{}}The user asks to see his personal information.\end{tabular}} \\ \hline
Precondition & \multicolumn{2}{l|}{\begin{tabular}[c]{@{}p{7.5cm}@{}}The chatbot is in a state in which the user is allowed to type messages independently, meaning that the user does not conduct another predefined conversation.\end{tabular}} \\ \hline
 & Step & Action \\ \cline{2-3} 
\multirow{-2}{*}{Flow of Events} & 1 & \begin{tabular}[c]{@{}p{6cm}@{}}The chatbot shows the user his stored interests and the current specified recommendation radius.\end{tabular}\\ \hline
 & Step & Action  \\ \cline{2-3}
\multirow{-2}{*}{Alternate Flow}& - & - \\ \hline
Postcondition & \multicolumn{2}{l|}{\begin{tabular}[c]{@{}p{7.5cm}@{}}The chatbot is in a state in which the user is allowed to type messages independently.\end{tabular}} \\ \hline
  & Step & Action \\ \cline{2-3} 
\multirow{-2}{*}{Exceptions}  & - & - \\ \hline
Frequency of Use & \multicolumn{2}{l|}{Low} \\ \hline
Importance & \multicolumn{2}{l|}{Low}  \\ \hline
\end{tabular}
\caption{UC-07 - Show User Information}
\label{userinfousecase}
\end{table}

