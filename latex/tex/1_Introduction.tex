\capitulo{1}{Introduction}

A chatbot is a computer program that interacts with its users through a conversational interface. They have been a topic of interest in the field of computer science for decades, yet, due to the rise of smart devices in the last couple of years, chatbots have received a great deal of new attention. The possibilities provided by the conversational interface were rediscovered and combined with the state of the art instant messaging methods. 

This project concentrates on this upcoming trend by developing a chatbot that can be accessed through an instant messaging platform, in this case the messenger Telegram. Using the messenger’s interface, the user interacts with the chatbot and is provided with tourist recommendations. The developed application consists of three principal components: 

\begin{enumerate}
\item The chatbot which forwards the user input to the natural language processing platform api.ai and interprets the response.
\item The recommendation system which computes the personalized recommendation based on the user’s preferences.
\item The geographic information database which makes the needed information for recommendation available.
\end{enumerate}

In the following paper, the developed application is presented by examining the main objectives, underlying theories and way of proceeding as well as describing encountered challenges in the course of the project. 
